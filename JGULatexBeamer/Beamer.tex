\documentclass[xcolor={usenames,dvipsnames}, 
	hyperref={
	colorlinks=true, 						% Internetseiten werden farblich hervorgehoben
	linkcolor=black, 						% Farbe f�r interne Referenzen
	urlcolor=black,							% Farbe f�r Links auf Webseiten
	citecolor=black,						% Farbe f�r Zitate \cite{<bibtexid>}
	pdfpagelabels=false,
	%pdfauthor={}, 
	%pdftitle={}				% Verewigt Author und Titel in PDF-Informationen
	},
	ignorenonframetext,			% Keine Nummerierung auf erster Folie
	compress					% Minimize navigation bars
]{beamer}
%===== Formatierungs-Packages =====
	\usepackage[T1]{fontenc}				% fontenc und inputenc erm�glichen
	\usepackage[latin1]{inputenc}			% Silbentrennung und Eingabe 
	%\usepackage[utf8]{inputenc}
	\usepackage{lmodern}					% Schrift wirkt (in pdf-Ausgabe) flie�ender
  							
%===== Sprach-Packages =====
	\usepackage[ngerman]{babel}			% Babel f�r diverse Sprachanpassungen, 
  										% z.B. Anf�hrungszeichen
%===== Interne Latex-Packages =====
	\usepackage{fixltx2e} 				% Verbessert einige Kernkompetenzen von LaTeX2e
  
%===== Mathe-Packages ======
	\usepackage{amsmath, amssymb, amsfonts} 	% Mathematische Features der American Mathematical
	\usepackage{cancel}
	\usepackage[output-decimal-marker={,},  	% Deutsche Dezimaltrennung mit Komma
		separate-uncertainty = true,		  	% Fehlerangabe: \SI{3(2)}{\tesla}
		per-mode=fraction,					  	% Einheiten als Bruch darstellen
		exponent-product = \cdot,			  	% Exponentialschreibweise mit Malzeichen \SI{3e8}{\tesla}
		math-ohm,
		range-phrase = -					% Option f�r Bereichsangabe \SIrange{3}{4}{\tesla}
		]{siunitx} 						  	
 								% Elementar wichtig f�r Einheiten \siunit{3}{\milli\meter}					% \unit{\tesla}, \num{<Zahl>}
  
%===== Grafik/Tabellen-Packages ======
	\usepackage{xcolor}						% Erlaubt das Verwenden von Farben
	\usepackage{graphicx} 					% Erlaub das Einbinden von Bildern
	\usepackage{multirow} 					% Erlaubt multicolumn{3}{c}{Bla}
	\usepackage{rotating} 					% Erlaub sidewaystable-Umgebung
	%\usepackage[miktex,subfolder,siunitx]{gnuplottex} % Gnuplot in Latex
  
%===== bibliography =====
	\usepackage[numbers,square]{natbib} 	%Einbinden der Bibliothek mit "[1]" Zitat
  
%===== Sonstiges ======
	\usepackage{url}						% Erlaubt das Einbinden einer URL
	\usepackage{pdfpages}					% \includepdf{<file>.pdf} wird verf�gbar
	\usepackage[flushmargin,				% Fu�noten b�ndig mit Seitenr�ndern
		hang,									% H�ngender Zeilenumbruch bei Fu�noten 
		bottom]{footmisc}						% Zwingt Fu�noten ans Ende der Seite
	\usepackage{hyperref} 	% Hyperref-Package verlinkt interne Referenzen
	\usepackage{lipsum} 	% F�r Testzecke \lipsum[1]
  
%===== Beamer template - Spezifikationen ======
	\usepackage{beamerthemetexsx}
	\setbeamertemplate{mini frames}[box]
	%\renewcommand{\bibsection}{\section{Literature}}

%==== Informationen ====
	
	\title[Kurztitel anstatt des langen Titels]{Ein sehr langer Titel f�r die Pr�sentation} 
	%\subtitle[Optionaler Subtitle]{Optionaler Subtitle des sehr langen Titels der Pr�sentation}
	\author[Kurzform Autorname]{Autorvorname Autornachname} 
	\institute[Kurzform Institut/Veranstaltung]{Institut an der JGU Mainz}
	\date{Tag.\,Monat.\,Jahr}
	%Titelgraphic
	\titlegraphic{\includegraphics[height=2.1cm,keepaspectratio]{Logos/Gefoerdert_vom_BMBF_weiss.pdf}}
	%Logo on each slide
	%\logo{
		%\vspace*{-.25cm} %vertical displacement of logo position
		%\includegraphics[height=1cm,keepaspectratio]{Logos/Gefoerdert_vom_BMBF_eng.jpg}
		%\hspace*{.25cm} %horizontal displacement of logo position
		%}
	%}
\pdfminorversion=4	% PDF 1.4, um Lesefehlern mit Acrobat entgegenzuwirken
	
%==== Begin presentation ====

\begin{document}

\begin{frame}[plain,noframenumbering]  %Keine Fu�zeile auf erster Seite und keine Nummerierung
	\titlepage
\end{frame}
	
\addtocounter{framenumber}{-1} %Titelseite wird nicht gez�hlt im Counter

%\setbeamertemplate{footline}{authorinstitutetitle} %Aktiviere Fu�zeile

%---------------------------------------------------------------------

\begin{frame}
	\frametitle{Inhaltsverzeichnis}
	\tableofcontents
\end{frame}%[pausesections]} 

%---------------------------------------------------------------------

\section{Titel der Section 1}

\frame{\sectionpage}

%---------------------------------------------------------------------

\subsection{Titel der Subsection 1}

\begin{frame}
	\subsectionpage
\end{frame}

\begin{frame}
	\frametitle{Folientitel: Aufz�hlungen} 
	\framesubtitle{Folienuntertitel: Nummeriert und nicht nummeriert.} 
	
	\begin{enumerate} 
	\item Item 1 
	\item Item 2 
	\item Item 3 
	\end{enumerate}
	\begin{itemize}
	\item Eins\dots
	\item Zwei\dots
	\end{itemize}
\end{frame}

%---------------------------------------------------------------------

\subsection{Subsection 2}

\begin{frame}
	\frametitle{Blocks und Theoreme}
	\framesubtitle{Block, alertblock, exampleblock,}
	\begin{columns}[t]
		\column{.5\textwidth}
			
				\begin{block}{Blocktitel}
						Blocktext
				\end{block}
				\begin{exampleblock}{Beispielblocktitel}
						Beispielblocktext
				\end{exampleblock}
				\begin{alertblock}{Warnungsblocktitel}
								Warnungsblocktext
				\end{alertblock}
		\column{.5\textwidth}
			\begin{theorem}
				Theoremtext
			\end{theorem} 
			
			\begin{definition}
        Definition
			\end{definition}
			
			\begin{lemma}
        Lemma
			\end{lemma}
	\end{columns}
\end{frame}

\subsubsection{Subsubsection 2.1}

%---------------------------------------------------------------------

\section{Section 2}

\begin{frame}
\frametitle{Lorem ipsum}
\lipsum[1]
\end{frame}

%---------------------------------------------------------------------


%---------------------------------------------------------------------

%\section{Literature}
\begin{frame}
\frametitle{Literature}
\nocite{*}
\bibliographystyle{alpha}
\bibliography{beamerbibliography}
\end{frame}

\end{document}